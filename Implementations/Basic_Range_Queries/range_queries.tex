%! TEX root = ../implementations.tex

\chapter{Basic range queries }

\section{Types of range queries}
Depending on the type of function whose value we have to calculate over
the given range:
\begin{itemize}
		\item \textit{Idempotent functions}. These functions fulfill the 
				following condition $f(f(x))=f(x)$. For instance, the maximum
				and minimum of a list of values are idempotent. 
				Furthermore, since the GCD and LCM can be seen as 
				the \say{maximum} and the \say{minimum} of the exponents in the
				prime factorization, it is only natural that they fulfill the same 
				property.

				The most important property of these functions for range queries
				is that we can evaluate the elements of the range several times 
				without affecting the result. For instance:
				\[
						\min(a_l,\dots ,a_r)= \min \left (
								\min (a_l, \dots,  a_k), \min (a_{k},\dots, a_r)
						\right ) \quad l < k <r
				\]
				even though the union of the intervals includes $a_k$ twice.
				This is the property in which sparse tables are based.


		\item  \textit{Reversible functions}. Functions that have an inverse
				also have advantages in range queries. That is because we
				can easily remove elements from any range to obtain a smaller
				one. In these cases we can obtain the final range by adding
				and subtracting ranges that have been precalculated.

				For instance, the addition, subtraction and xor operations have 
				an inverse (subtraction, addition and xor respectively). 

\end{itemize}

\newpage
There are also other problems that can be transformed into a range query easily.

\subsection{LCA using RMQ}
To use this algorithm we need a rooted tree with undirected edges where the 
nodes are labeled in a particular order. Specifically, nodes at a greater
depth have to be labeled with a greater value. Furthermore, the algorithm
could be extended to process a forest (instead of a single tree) but 
this version cannot process forests.

We can traverse the tree using dfs and store in an array each node that we 
encounter. If we also keep track of the time at which each node was found
for the first time, the LCA can be converted into calculating the minimum
in an array of length $E+V$.

\begin{figure}[h!]
		\centering
		\resizebox{70 mm}{!}{\subimport{figures/}{LCA_to_RMQ.pdf_tex}}
\end{figure}
After running a DFS in the tree, we would get the following array of 
values and discovery times:
\begin{figure}[h!]
		\centering
		\begin{subfigure}{0.55\textwidth}
		\centering
\begin{tikzpicture}
		\draw (-6,0) pic[]{array_rep={}{1, 2, 4, 
				2, 5, 2, 1, 3, 6, 8, 6, 3, 7, 3, 1}};
\end{tikzpicture}
		\end{subfigure}
		\begin{subfigure}{0.4\textwidth}
		\centering
		\scalebox{0.9}{
				\begin{tabular}{c|c||c|c}
						\footnotesize{ \textbf{Node}} & \footnotesize{\textbf{Time}}&
						\footnotesize{\textbf{Node}} & \footnotesize{\textbf{Time}}\\
						\hline \hline
						1 & 0 &5 & 4\\
						2 & 1 &6 & 8\\
						3 & 7 &7 & 12\\
						4 & 2 &8 & 9\\
				\end{tabular}
		}
		\end{subfigure}

\end{figure}

Now if we want to get the LCA between nodes 5 and 6, we have to obtain the minimum
value in the range [4, 8]. As we can see, the result is obviously 1.

If we compare this approach to binary lifting, we observe that we have a considerably
more expensive precalculation phase but, afterwards, answering any queries takes
constant time instead of logarithmic time. In particular, the complexities for this 
method are:
\begin{itemize}
		\setlength{\itemsep}{1pt}
		\item Precomputate the values: $\mathcal{O}(E\log(E))$
		\item Calculate the LCA of a node: $\mathcal{O}(1)$
		\item Memory usage: $\mathcal{O}(E\log(E))$
\end{itemize}
Where $E=\text{edges}$.

\newpage


\section{Sparse table}
The sparse table is a data structure that calculates the value of 
a function over a range of immutable elements with the following complexity:
\begin{itemize}
		\setlength{\itemsep}{2 pt}
		\item \tbf{$\mathcal O(n\log n)$} to build the data structure
		\item \tbf{$\mathcal O(1)$} to answer queries on a range. 
		\item \tbf{$\mathcal O(n\log n)$} memory
\end{itemize}
Furthermore, the function must be \textbf{idempotent}. This allows the use of
overlapped intervals in the calculations. A Sparse Table can be adapted to 
answer queries with more general functions. However, it would lose the constant
time query capabilities and would be no more efficient than a segment tree or a BIT.


On the other hand, the data structure is rather simple and its implementation 
is considerable shorter than any other alternatives. 

The table consists of $\log n$ rows. Each one of them contains the value of the
function in intervals of length $2^j$ where $j$ is the row index. In particular,
table$[j][i]=f([i,i+2^j])$, where $f(I)$ is the value of $f$ over all the 
elements in $I$.

To construct the table we take advantage of the fact that the intervals of each
row are the exact union of two intervals of the previous row. Additionally, we 
have to create a lookup table for logarithms so that the queries can be 
processed fast enough.
It is important to notice that if many sparse tables where to be constructed,
it would be advisable to share the log array among them by populating it with
all the values that might be needed initially or using a DP recursive approach.

\newpage
\subsection{Example}
One of the best applications of the sparse table is to calculate the maximum of 
any range in an array of elements. The following illustration displays all the 
intervals that will be included in the sparse table. To avoid intersections, 
intervals that belong to the same level of the table may be displayed above 
or below each other.

In each interval, we can see both the index (top left corner) and the value of
the maximum in that interval. It can also be seen that all intervals can 
be calculated using the values obtained in the previous level.
\begin{figure}[h!]
		\centering

\begin{tikzpicture}
		\begin{scope}
				\foreach \m/\c in{
						0/1,
						1/3,
						2/4,
						3/8,
						4/6,
						5/1,
						6/4,
						7/2
				}{
				\pgfmathtruncatemacro{\om}{\m}
				\pgfmathsetmacro{\sm}{1*\m}
				\pgfmathsetmacro{\em}{1+\sm}
				\node[fit={(\sm,0.9)(\em,1.1)}, inner sep = 0 pt, label = center: \scalebox{0.7}{\om}](A){};
				\node[fit={(\sm,0.25)(\em,0.75)}, inner sep = 0 pt, label = center: \c](A){};
				}
		\end{scope}
		\begin{scope}
				\foreach \i/\c in 
				{
						0/1,
						1/3,
						2/4,
						3/8,
						4/6,
						5/1,
						6/4,
						7/2
				}
				{
				\pgfmathsetmacro{\sm}{0.1+1*\i}
				\pgfmathsetmacro{\lm}{0.1+1*\i+0.2}
				\pgfmathsetmacro{\em}{0.8+\sm}
				\node[fit={(\sm,-0.9)(\lm,-0.5)}, inner sep = 0 pt, ](A){\scalebox{0.5}{\i}};
				\node[draw, fit={(\sm,-1)(\em,-0.5)}, inner sep = 0 pt, label = center: \c](A){};
				}

				\foreach \i/\c/\m in 
				{
						0/3/0,
						1/4/1,
						2/8/0,
						3/8/1,
						4/6/0,
						5/4/1,
						6/4/0
				}
				{
				\pgfmathsetmacro{\sm}{0.1+1*\i}
				\pgfmathsetmacro{\lm}{0.1+1*\i+0.2}
				\pgfmathsetmacro{\em}{1.8+\sm}
				\pgfmathsetmacro{\sh}{-2.25-\m*0.75}
				\pgfmathsetmacro{\lh}{0.1+\sh}
				\pgfmathsetmacro{\eh}{0.5+\sh}

				\node[fit={(\sm,\lh)(\lm,\eh)}, inner sep = 0 pt, ](A){\scalebox{0.5}{\i}};
				\node[draw, fit={(\sm,\sh)(\em,\eh)}, inner sep = 0 pt, label = center: \c](A){};
				}

				\foreach \i/\c/\m in 
				{
						0/8/0,
						1/8/1,
						2/8/2,
						3/8/3,
						4/6/0
				}
				{
				\pgfmathsetmacro{\sm}{0.1+1*\i}
				\pgfmathsetmacro{\lm}{\sm+0.2}
				\pgfmathsetmacro{\em}{3.6+\sm}
				\pgfmathsetmacro{\sh}{-4.5-\m*0.75}
				\pgfmathsetmacro{\lh}{0.1+\sh}
				\pgfmathsetmacro{\eh}{0.5+\sh}

				\node[fit={(\sm,\lh)(\lm,\eh)}, inner sep = 0 pt, ](A){\scalebox{0.5}{\i}};
				\node[draw, fit={(\sm,\sh)(\em,\eh)}, inner sep = 0 pt, label = center: \c](A){};
				}

				\foreach \i/\c/\m in 
				{
						0/8/0
				}
				{
				\pgfmathsetmacro{\sm}{0.1+1*\i}
				\pgfmathsetmacro{\lm}{\sm+0.2}
				\pgfmathsetmacro{\em}{7.2+\sm}
				\pgfmathsetmacro{\sh}{-8.25-\m*0.75}
				\pgfmathsetmacro{\lh}{0.1+\sh}
				\pgfmathsetmacro{\eh}{0.5+\sh}

				\node[fit={(\sm,\lh)(\lm,\eh)}, inner sep = 0 pt, ](A){\scalebox{0.5}{\i}};
				\node[draw, fit={(\sm,\sh)(\em,\eh)}, inner sep = 0 pt, label = center: \c](A){};
				}

				\node[anchor=east] (a) at (-0.5,0.45){\texttt{array:}};
				\node[anchor=east] (a) at (-0.5,-0.75){\texttt{table[0]:}};
				\node[anchor=east] (a) at (-0.5,-2.5){\texttt{table[1]:}};
				\node[anchor=east] (a) at (-0.5,-5.5){\texttt{table[2]:}};
				\node[anchor=east] (a) at (-0.5,-8){\texttt{table[3]:}};

		\end{scope}
		\end{tikzpicture}
\end{figure}

\noindent
Now, if we want to calculate the maximum between indices 2 and 7 we will:
\begin{enumerate}
		\setlength{\itemsep}{2pt}
		\item Calculate the length of the interval. In this case, $7-2+1=6$
		\item Calculate the maximum value of $k$  such that $2^k<6$. In this
				case, $k=2$.
		\item Pick the interval that starts on 2 with length $2^k=4$ and the
				interval that starts on $7-2^k+1$ with the same length. 
				That is to say, intervals 2 and 4 from level 2 of the table.

				As we can see these intervals cover the entire range, albeit 
				they overlap. However, the overlap does not affect the result
				of the maximum.
		\item Take the maximum between the values of the two intervals 
				selected: $\max(8,6)=8$
\end{enumerate}

\newpage
\cppcode[firstline=20]{code/SparseTable.cpp}
\newpage

\subsection{Sparse Table {\scriptsize(sort of)} with $\mathcal{O}(n)$ memory}

This variant of the sparse table is only capable of answering very
specific queries. In particular, the operation must meet the following
constraints:
\begin{itemize}
		\setlength{\itemsep}{2pt}
		\item It is idempotent (as in a normal Sparse Table)
		\item The result of the operation over a range is an element
				in that range
\end{itemize}
Therefore, this data structure will mostly only be used for querying the
maximum or the minimum over a range. On the other hand, we obtain a data 
structure that uses significantly less memory and can be built in
considerably less time, albeit with slower queries.
\subsection*{Explanation}
With that goal in mind, we will group the elements of the array in blocks.
These blocks should be of size $\log (n)$. However, since $n$ will never
be larger than $2^{30}$, we can set their size to 30. Now we we will build 
two data structures:
\begin{itemize}
		\setlength{\itemsep}{2pt}
		\item A Sparse Table where each element is a block of the array. 
				In order to simplify this task, in the Sparse Table we will
				store the index of the result of applying the operation to the 
				range instead of the result itself. 

				Therefore, the elements in the first row of the table will be the 
				index of the result of applying the operation to each block of the array while
				the elements in the following layers will represent applying the
				operation to 2, 4, 8 \dots blocks.

		\item A structure that stores the result for every range of 30 contiguous 
				elements in the array.
\end{itemize}

If we have these two data structures we can split the (large) query into three subqueries
that will most likely overlap. In particular, we will split the range $[l, r]$ into:
\begin{itemize}
		\setlength{\itemsep}{0pt}
		\item A range of size 30 starting at $l$.
		\item The blocks that are fully contained within the interval.
		\item A range of size 30 that ends at $r$.
\end{itemize}
The middle range will be processed using the Sparse Table of blocks while the other 
two must be calculated using the new data structure.

\begin{figure}[h!]
		\centering
		\resizebox{0.75\textwidth}{!}{\subimport{figures/}{Optimized_Sparse.pdf_tex}}
\end{figure}
\newpage
In order to answer the small queries, we will use a bitmask for every position. That
is to say, a vector of $n$ integers where each of them is a bitmask for that position.
Let's look at the bitmask of position $r$. The bit $r-k, \ (k<30)$ will 
be one iff that element  is the result in the range $[r-k, r]$.  
For instance, at position $6$ of this example array the bitmask would be:
\begin{figure}[h!]
\centering
\begin{tikzpicture}
		\begin{scope}[]
				\foreach \m/\c/ \v in{
						0/1/1,
						1/4/0,
						2/2/1,
						3/6/0,
						4/7/0,
						5/4/1,
						6/5/1
				}{
				\pgfmathtruncatemacro{\om}{\m}
				\pgfmathsetmacro{\sm}{0.5*\m}
				\pgfmathsetmacro{\em}{0.5+\sm}
				\node[draw, fit={(\sm,0)(\em,0.5)}, inner sep = 0 pt, label = center: \c](A){};
				\node[fit={(\sm,0.6)(\em,0.8)}, inner sep = 0 pt, label = center: \scalebox{0.7}{\om}](A){};
				\node[fit={(\sm,-1)(\em,0)}, inner sep = 0 pt, label = center: \scalebox{0.9}{\v}](A){};
				}
		\end{scope}
		\node[anchor=east] (a) at (-0.5,0.25){\texttt{array:}};
		\node[anchor=east] (a) at (-0.5,-0.5){\texttt{mask:}};
\end{tikzpicture}
\end{figure}
This bitmask approach is particularly useful because the vector of bitmasks can be built in linear time. 
If we look at the code below (\texttt{build\_mask()}), the \texttt{while} loop 
will only be executed at most once for every element while 
the \texttt{for} loop iterates through all elements exactly once.

\cppcode[firstline=20]{code/Linear_RMQ.cpp}




\newpage
\section{Fenwick tree (BIT)}
A Fenwick tree or binary indexed tree (BIT) can be used to calculate the value
of a \textbf{reversible} function $F$ over a range of values. Given an array of elements
$A$ of size $N$:
\begin{itemize}
		\setlength{\itemsep}{2pt}
		\item Given a range $[l,r]$, the value $F(a_l,\dots, a_r)$ can be calculated
				in \tbf{$\mathcal{O}(\log n)$}
		\item Updating one of the values of the array takes \tbf{$\mathcal{O}(\log n)$}
		\item Requires the same amout of memory as the array $A$
\end{itemize}
To do so, we create a new array \texttt{ft} where each element represents
the value of the function over a particular range. Let LSOne($i$) be the 
result of setting to 0 all bits of $i$ except for the least significant one.
For instance, LSOne$(0110)=0010$. Then, the $i$-th element of \texttt{ft} 
contains the value $F(a_{i-\text{LSOne}(i)+1},\dots, a_{i})$. 

In most cases, $F$ will be the sum of all elements $F(a_l,\dots, a_r)=\sum_l^r{a_i}$.
In that case, \text{ft[i]} contains the sum of all the elements from 
position $i-\text{LSOne}(i)$ (excluded) to position $i$ (included).


It is also important to note that this BIT is \textbf{1-indexed interally} to simplify slightly 
the implementation. However, the interface automatically subtracts one to all index parameters
given.
\subsection*{Range Update and Point Query}
The standard BIT allows point update and range query. Using a difference array, we can 
transform them into range update and point query. If $B$ is the difference array
of $A$, the elements of $B$ are defined as:  $b_i=a_i- a_{i-1}$ and $b_1=a_1$.
Therefore:
\begin{itemize}
		\setlength{\itemsep}{2pt}
		\item $\sum_{1}^i b_j=a_i$, converting the range query into 
				a point query
		\item To update the elements in the range $[l,r]$ by $d$, we can add
				$d$ to the $l$-th element and subtract $d$ from the $r$-th element.
				In doing so, queries that reach until a point
				between $l$ and $r$ will be increased by $d$ but queries that reach 
				further will not be affected.
				Therefore we have converted two point updates into a range update.

\end{itemize}

\subsection*{Example}
For instance, if we use the addition operation, array on the left would be processed
into the BIT tree displayed on the right. Each box under the BIT tree represents
the elements that are added to get the value of that position.
\begin{figure}[h!]
		\centering
		\scalebox{0.85}
		{
\begin{tikzpicture}
		\begin{scope}[shift={(-6,0)}]
				\foreach \m/\c in{
						0/1,
						1/3,
						2/4,
						3/8,
						4/6,
						5/1,
						6/4,
						7/2
				}{
				\pgfmathtruncatemacro{\om}{\m}
				\pgfmathsetmacro{\sm}{0.5*\m}
				\pgfmathsetmacro{\em}{0.5+\sm}
				\node[draw, fit={(\sm,0)(\em,0.5)}, inner sep = 0 pt, label = center: \c](A){};
				\node[fit={(\sm,0.6)(\em,0.8)}, inner sep = 0 pt, label = center: \scalebox{0.7}{\om}](A){};
				}
		\end{scope}
		\begin{scope}
				\foreach \m/\c in{
						0/1,
						1/4,
						2/4,
						3/16,
						4/6,
						5/7,
						6/4,
						7/29
				}{
				\pgfmathtruncatemacro{\om}{\m+1}
				\pgfmathsetmacro{\sm}{0.5*\m}
				\pgfmathsetmacro{\em}{0.5+\sm}
				\node[draw, fit={(\sm,0)(\em,0.5)}, inner sep = 0 pt, label = center: \c](A){};
				\node[fit={(\sm,0.6)(\em,0.8)}, inner sep = 0 pt, label = center: \scalebox{0.7}{\om}](A){};
				}
				\draw (0, -1) rectangle ++(0.5,0.3);
				\draw [->] (0.25,-0.7)--(0.25,0);
				\draw (1, -1) rectangle ++(0.5,0.3);
				\draw [->] (1.25,-0.7)--(1.25,0);
				\draw (2, -1) rectangle ++(0.5,0.3);
				\draw [->] (2.25,-0.7)--(2.25,0);
				\draw (3, -1) rectangle ++(0.5,0.3);
				\draw [->] (3.25,-0.7)--(3.25,0);

				\draw (0, -1.7) rectangle ++(1.0,0.3);
				\draw [->] (0.75,-1.4)--(0.75,0);

				\draw (2, -1.7) rectangle ++(1.0,0.3);
				\draw [->] (2.75,-1.4)--(2.75,0);

				\draw (0, -2.4) rectangle ++(2.0,0.3);
				\draw [->] (1.75,-2.1)--(1.75,0);

				\draw (0, -3.1) rectangle ++(4.0,0.3);
				\draw [->] (3.75,-2.8)--(3.75,0);

		\end{scope}
\end{tikzpicture}
}
\end{figure}

\newpage
\cppcode[firstline=20]{code/BIT.cpp}

