%! TEX root = ../implementations.tex
\chapter{Geometry}

\section{Area of polygon}

\begin{wrapfigure}{r}{0.5\linewidth}
	\centering
	\resizebox{80 mm}{!}{\subimport{Figures/}{geometry_area.pdf_tex}}
	\vspace*{-40 pt}
\end{wrapfigure}

In this problem, we are given a polygon expressed as a list of vertices.
We are tasked with calculating the area of the polygon.

For now, we will assume that the points are ordered in a clock-wise fashion
and we will consider three different scenarios for each segment:
\begin{itemize}
	\item $\Delta x=0$. It does not add any area to the polygon.
	\item $\Delta x\ne 0$. It has an area associated with it that is equal
		to the area between the segment and the $x$ axis
		\begin{itemize}
			\item $\Delta x > 0$. We consider this area positive
				(green on the figure)
			\item $\Delta x < 0$. We consider this area negative
				(red on the figure)
		\end{itemize}
\end{itemize}

Now we add all the areas associated with each segment and we get a total area.
If the vertices were ordered in a counter-clock-wise fashion, this total
area will be negative. Therefore, we will only consider the absolute
value of the result.
\cppcode[firstline=20,lastline=63]{code/geometry/polygon_area.cpp}
\noindent \textbf{\boldmath Running time: $\mathcal{O}(n)$}
\\ {\small (n = vertices)}



