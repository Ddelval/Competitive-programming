%! TEX root = ../implementations.tex
\chapter{Graphs}

\section{Dijkstra's}
Shortest path from \texttt{orig} node to \texttt{dest} (or to every node) in a graph
that does not contain negative edges. 
It chooses the best path greedily in each iteration and, therefore, it only works
on graphs without negative weights. 
\cppcode[firstline=20]{code/dijkstra.cpp}
\noindent \textbf{\boldmath Running time: $\mathcal{O}((E+V)\log V)$}
\\ {\small(V = vertices, E = edges)}
\subsection*{Remarks}
\begin{itemize}
	\item If we ignore the check in line 30, we can return the distances 
		vector, which will contain the shortest distance from \texttt{dist}
		to every other node.
	\item If we are doing some kind of pruning it is imperative that we prune 
	as many branches as possible in the main loop. That is to say, we should
	introduce as many \texttt{if} statements in line 36 to make sure that we
	run the \texttt{for} loop as few times as possible. 

	An example of this approach is problem \texttt{UVA-11635}. In that problem, we
	add a lot of branches to the queue (we may run the \texttt{for} loop
	twice in some nodes) but we prune them in the main loop. Thus the 
	running time is still acceptable.

\item  However, pruning the branches with a higher cost can be quite complicated 
	when we have to optimize several factors (see section \ref{graph:dijkstra:distances}).
	Therefore, we will omit the check in line \texttt{33} if we think it might discard 
	relevant options.

\end{itemize}

\newpage
\subsection{Dijkstra's with cost and distance}
\label{graph:dijkstra:distances}
In this problem we are going to define two values for each edge, cost and distance:
\begin{itemize}
	\setlength\itemsep{0 pt}
	\item \textit{Cost}. This is the number that we want to minimize. It is the 
		equivalent to the usual cost in a normal Dijkstra problem
	\item \textit{Distance}. We define the distance of a path as the sum of the
		distances of the edges in it. And the distance of the path from source
		to destination cannot exceed a given limit (\texttt{B})
\end{itemize}
This problem can be solved with a slightly modified version of Dijkstra's:
\cppcode[firstline=20]{code/dijkstra_2.cpp}
\noindent \textbf{\boldmath Running time: $\mathcal{O}\big(B\cdot(V+E\log E)\big)$}
\\ {\small(V = vertices, E = edges, B = max\_distance)}

In this algorithm, we will process each node at most \texttt{B} times. Furthermore, 
if we process a node with a cost $c$ and a distance $d$, we know that this is the 
best possible cost for that distance $d$ since all the nodes in the queue have a higher
cost already. Therefore, we can close the node for that particular distance value (line 37).

An example of this algorithm is \texttt{SWERC-19\_20-A}



\section{Bellman Ford's}
Shortest parth from \texttt{orig} to every other node. It is slower than Dijkstra but 
it works on graphs with negative weights. 

This algorithm works by trying to relax every edge $V-1$ times. If there are no 
negative cycles, after $V-1$ iterations, we must have found the minimum distance
to every node. Therefore if after these iterations, we run another
iteration and the distance to a node decreases, we must have a negative cycle.

\cppcode[firstline=20]{code/bellman_ford.cpp}
\noindent \textbf{\boldmath Running time: $\mathcal{O}(VE)$}
\\ {\small(V = vertices, E = edges)}
\subsection*{Remarks}
\begin{itemize}
	\item If we keep track of the distance that decrease when we check for
		a negative cycle, we will get at least one node of 
		every negative cycle present in the graph.

		We can use this, for instance, to check if we can reach a node with 
		a cost smaller than a given bound. If it is connected to a node in a 
		negative cycle, it's distance will be as small as we want it to be
		(by looping in the cycle).

		This can be seen at play in \texttt{UVA-10557}
		\newpage
	\item If we modify slightly the main loop, iteration $i$ will be the 
		result of considering paths of at most $i+1$ edges:
		\begin{minted}{cpp}
for (int i = 0; i < n - 1; ++i) {
        for (auto e : edges) {
            dists2[e.fi.se] = min(dists2[e.fi.se], dists[e.fi.fi] + e.se);
        }
	dists = dists2;
}
		\end{minted}
		This can be seen at play in \texttt{UVA-11280}
		

\end{itemize}	

\section{Warshall's}
Warshall's algorithm solves the APSP (All pairs shortest path) using DP.
Each iteration of the outer loop tries to add a node (the $k$-th node in
particular) to the path between all pairs of nodes. 
We can think of the comparison as: 
\begin{center}
		\textit{ Is the path from $i$ to $j$ shorter if we first 
		move from $i$ to $k$ and, then, from $k$ to $j$? }
\end{center}
It is important to note that we must use an adjacency matrix in this
implementation.
\cppcode[firstline=20]{code/basicWarshall.cpp}
\noindent \textbf{\boldmath Running time: $\mathcal{O}(V^3)$}
\\ {\small (V = vertices, E = edges)}
\subsubsection*{Obtaining the path}
To get the explicit path, we will store the last vertex in the 
path that the algorithm found to go from $i$ to $j$. That is to say,
after  a given iteration the optimal path from $i$ to $j$ is:
\begin{figure}[h!]
		\centering
\begin{tikzpicture}
	\begin{scope}[every node/.style = {circle, thick, draw}]
		\node (A) at (0,0) {$i$};
		\node (C) at (5,0) {$j$};
	\end{scope}
	\begin{scope}[every node/.style = {scale=0.75, circle,   draw}]
		\node (B1) at (1,0) {$a_1$};
		\node (B2) at (2,0) {$a_2$};
		\node (E2) at (4,0) {$a_{l}$};
	\end{scope}
	\begin{scope}[every node/.style = { circle   }]
		\node (D) at (3,0) {$\dots$};
	\end{scope}
	\begin{scope}[>={Stealth[black]},
			every edge/.style={draw=black, very thick}]
		\path [-] (A) edge  (B1);
		\path [-] (B1) edge  (B2);
		\path [-] (B2) edge  (D);
		\path [-] (D) edge  (E2);
		\path [-] (E2) edge  (C);
	\end{scope}
\end{tikzpicture}
\end{figure}
Then we should store that the last node in the path that goes from $i$
to $j$ is $a_j$. 
The code to do so is the following:
\cppcode[firstline=20]{code/pathWarshall.cpp}
\newpage




\section{DFS}
\subsection{Articulation points and bridges}
These algorightms can be used in undirected graphs, and we
will use the following definitions:
\begin{itemize}
	\def\itemsep{0 pt}
	\item \textbf{Articulation point}. A node whose removal would increase the number
		of connected components of the graph. That  is to say
		that it \say{splits} a connected component.
	\item \textbf{Bridge}. An edge whose removal increases the number of 
		connected components in the graph.
\end{itemize}
We will use a modified version of DFS to solve this problem. We mainly introduce two new
properties for every node:
\begin{itemize}
	\def\itemsep{0 pt}
	\item \texttt{num}. Time at which the node was first explored by DFS
	\item \texttt{low}. Earliest node that can be found in the DFS spanning 
		tree that starts from this node
\end{itemize}
When we visit a node, for every edge, there are two options:
\begin{itemize}
\def \itemsep{0pt}
	\item \textbf{Tree edge}. This edge point to a node that has not been 
		discovered yet. As such we explore it (calling \texttt{dfs})
		and, we update the value of \texttt{low} for the current
		node. 

		After the update, we can now process the child since we will
		not visit it again and it's DFS tree has been fully explored

	\item \textbf{Back edge}. This edge points to a node that has already
		been visited. Therefore, it will have a relatively low 
		\texttt{num} and we use it to update the \texttt{low}
		value of the current node
\end{itemize}
\newpage
\cppcode[firstline=20]{code/articulationPts.cpp}
\noindent \textbf{\boldmath Running time: $\mathcal{O}(V+E)$}
\\ {\small (V = vertices, E = edges)}
\subsubsection*{Explanation}

The first graph that we will consider is the following:
\begin{figure}[h]
\centering
\begin{tikzpicture}
	\begin{scope}[every node/.style = {circle, thick, draw}]
		\node (A) at (0,0) {A};
		\node (B) at (3,0) {B};
		\node (C) at (6,0) {C};
	\end{scope}
	\begin{scope}[>={Stealth[black]},
			every edge/.style={draw=black, very thick}]
		\path [-] (A) edge  (B);
		\path [-] (B) edge  (C);
	\end{scope}
\end{tikzpicture}
\end{figure}
After applying DFS on $A$ we get the following DFS spanning tree.
Above every node, we have included \texttt{num / low}.
\begin{figure}[h]
\centering
\begin{tikzpicture}
	\begin{scope}[every node/.style = {circle, thick, draw},
		every label/.append style={font = \small, yshift = -0.5 em}]
		\node[label={1/1}] (A) at (0,0) {A};
		\node[label={2/2}](B) at (3,0) {B};
		\node[label={3/3}](C) at (6,0) {C};
	\end{scope}
	\begin{scope}[>={Stealth[black]},
			every edge/.style={draw=black, very thick}]
		\path [->] (A) edge  (B);
		\path [->] (B) edge  (C);
	\end{scope}
\end{tikzpicture}
\end{figure}

This is a rather simple graph and it's only articulation point is $B$. This is because
it has a child whose \texttt{low} value is greater than or equal to $B$'s  \texttt{num}
value. Therefore, there is no connection from that child (namely $C$) to a node explored
before $B$ that does not go through $B$. 

However, we can already see that the root has to be treated as a separate case. 
The root will be an articulation point iff it has more than one child in its DFS tree. 
It is important to note that the children of the DFS tree need not be the same as the 
children of the root in the initial graph.

Let's look at another example
\begin{figure}[h]
\centering
\begin{tikzpicture}
	\begin{scope}[every node/.style = {circle, thick, draw},
		every label/.append style={font = \small, yshift = -0.5 em}]
		\node (A) at (0,0) {A};
		\node (B) at (3,0) {B};
		\node (C) at (6,0) {C};
		\node (D) at (3,-2) {D};
	\end{scope}
	\begin{scope}[>={Stealth[black]},
			every edge/.style={draw=black, very thick}]
		\path [-] (A) edge  (B);
		\path [-] (B) edge  (C);
		\path [-] (C) edge  (D);
		\path [-] (D) edge  (A);
	\end{scope}
\end{tikzpicture}
\end{figure}

As before, we show the DFS spanning tree:
\begin{figure}[h!]
\centering
\begin{tikzpicture}
	\begin{scope}[every node/.style = {circle, thick, draw},
		every label/.append style={font = \small, yshift = -0.5 em}]
		\node [label={1/1}] (A) at (0,0) {A};
		\node [label={2/1}](B) at (3,0) {B};
		\node [label={3/1}](C) at (6,0) {C};
		\node [label={4/1}](D) at (3,-2) {D};
	\end{scope}
	\begin{scope}[>={Stealth[black]},
			every edge/.style={draw=black, very thick}]
		\path [->] (A) edge  (B);
		\path [->] (B) edge  (C);
		\path [->] (C) edge  (D);
		\path [->] (D) edge  (A);
	\end{scope}
\end{tikzpicture}
\end{figure}

Now we have no articulation point since there is no node that has a child
with a \texttt{low} greater than or equal  than the parent's \texttt{num}. 
This difference is caused by the fact that now there is an edge that makes $C$
accessible through a path that does not involve traversing $B$.

In this case, we can see how the root has only one child in the spanning tree
but two in the initial graph. This reflects the fact that those numbers need not
coincide.

Finally, the condition for a bridge is: \texttt{low[child]>num[parent]}. This is 
equivalent to  stating that the child has no other way of reaching either the
parent or a node that was explored before the parent

\subsection{Trajan's algorithm for strongly connected components}
In a directed graph, we say that a subset of vertices comprises a strongly 
connected component if every vertex is reachable from every other vertex in
this subset.

We will now present an algorithm that divides the graph into strongly
connected parts that are as large as possible. It will use some of the 
same concepts as in the previous section. 
\subsubsection*{Explanation}
The main idea is the following: if when we are done with a node, we have not been
able to reach a node that was further back, this one is the root of a SCC. 

Let's explain this reasoning with an example:

\begin{figure}[h]
\centering
\begin{tikzpicture}
	\begin{scope}[every node/.style = {circle, thick, draw},
		every label/.append style={font = \small, yshift = -0.5 em}]
		\node (A) at (0,0) {A};
		\node (B) at (3,0) {B};
		\node (C) at (3,-2) {C};
		\node (D) at (6,0) {D};
		\node (E) at (9,0) {E};
		\node (F) at (12,0) {F};
		\node (G) at (9,-2) {G};
		\node (H) at (12,-2) {H};
	\end{scope}
	\begin{scope}[>={Stealth[black]},
			every edge/.style={draw=black, very thick}]
		\path [->] (A) edge  (B);
		\path [<-] (B) edge  (C);
		\path [<-] (C) edge  (D);
		\path [->] (B) edge  (D);
		\path [->] (D) edge  (E);
		\path [->] (E) edge  (F);
		\path [->] (F) edge  (H);
		\path [->] (H) edge  (G);
		\path [->] (G) edge  (E);
	\end{scope}
\end{tikzpicture}
\end{figure}
And, after running DFS, we would get the following 
spanning tree with three SCCs.

\begin{figure}[h]
\centering
\begin{tikzpicture}
	\begin{scope}[every node/.style = {circle, thick, draw},
		every label/.append style={font = \small}]
		\node[label={0/0},draw=red] (A) at (0,0) {A};
		\node[label={1/1},draw=red] (B) at (3,0) {B};
		\node[label=below:{3/1}] (C) at (3,-2) {C};
		\node[label={2/1}] (D) at (6,0) {D};
		\node[label={4/4},draw=red] (E) at (9,0) {E};
		\node[label={5/4}] (F) at (12,0) {F};
		\node[label=below:{7/4}] (G) at (9,-2) {G};
		\node[label=below:{6/4}] (H) at (12,-2) {H};
	\end{scope}
	\begin{scope}[>={Stealth[black]},
			every edge/.style={draw=black, very thick}]
		\path [->] (A) edge  (B);
		\path [<-] (B) edge  (C);
		\path [<-] (C) edge  (D);
		\path [->] (B) edge  (D);
		\path [->] (D) edge  (E);
		\path [->] (E) edge  (F);
		\path [->] (F) edge  (H);
		\path [->] (H) edge  (G);
		\path [->] (G) edge  (E);
	\end{scope}
\draw[rounded corners=15pt] (-1,-3.5) rectangle ++(2,5.3);
\draw[rounded corners=15pt] (2,-3.5) rectangle ++(5,5.3);
\draw[rounded corners=15pt] (8,-3.5) rectangle ++(5,5.3);
\end{tikzpicture}
\end{figure}

Let's look  at the moment in which we close node $F$. As we can see the \texttt{low}
value is lower than the \texttt{num}. Therefore, there is a connection from $F$ to
another node that was visited before and it will be a part of the SCC \say{generated}
by that node.

However, when we close $E$, we can see that the \texttt{low} and the \texttt{num}
are equal. As a result, there is no way to get to a node that has a number lower than
E's through the spanning tree of $E$. Therefore, there is no way to have a bigger 
SCC that contains $E$.

To get the nodes that are part of the SCC, we need to get all the nodes in the stack before
the one that \say{generates} the SCC ($g$). This is because the stack only contains the 
nodes that belong to the spanning tree of $g$ and do not belong to any other SCC. 
Therefore, they must belong to the one generated by $g$. 

Furthermore, all those nodes will have the same \texttt{low} value since the \texttt{low}
any node with a higher value must have been processed before. Therefore, $g$ is accessible
from all those nodes and, clearly, all those nodes are accessible from $g$.
Thus, they fulfill the definition of an SCC.



\newpage
\cppcode[firstline=20]{code/trajanSCC.cpp}
\noindent \textbf{\boldmath Running time: $\mathcal{O}(V+E)$}
\\ {\small (V = vertices, E = edges)}

\section{Kosaraju's}
Kosaraju's algorithm is a slightly simpler method for finding the SCC of a graph. 
On the other hand, it is also somewhat slower than Trajan's approach since it will
require running DFS on the graph twice. The algorithm follow this procedure:
\begin{enumerate}
	\item Run DFS on the graph and store the vertices in postorder in a list 
		called \texttt{postorder}.
	\item Reverse the graph
	\item Loop through the nodes in \texttt{postorder} starting from the one that 
		was closed the last. For each of them check if it has already been 
		added to a SCC (visited). If it hasn't, run DFS from that node and
		create a SCC with all the nodes that this DFS visits and hadn't been
		visited before.
\end{enumerate}
\subsubsection{Example}
Let's look at an example graph where we have run DFS starting first in node 
$A$ and, since it didn't explore the entire graph, we also run it from node $C$.
We have included the time at which each node was closed:
\begin{figure}[h]
\centering
\begin{tikzpicture}
	\begin{scope}[every node/.style = {circle, thick, draw},
		every label/.append style={font = \small}]
		\node [label=8] (A) at (0,0) {A};
		\node [label=7] (B) at (3,0) {B};
		\node [label=14] (C) at (6,0) {C};
		\node [label=below:5] (D) at (0,-2) {D};
		\node [label=below:6] (E) at (3,-2) {E};
		\node [label=below:12] (F) at (6,-2) {F};
		\node [label=13] (G) at (9,-1) {G};
	\end{scope}
	\begin{scope}[>={Stealth[black]},
			every edge/.style={draw=black, very thick}]
		\path [->] (A) edge[bend right=30] (B);
		\path [->] (A) edge (D);
		\path [->] (B) edge[bend right=30] (A);
		\path [->] (B) edge (E);
		\path [->] (C) edge (B);
		\path [->] (C) edge (G);
		\path [->] (F) edge (C);
		\path [->] (F) edge (E);
		\path [->] (G) edge (F);
		\path [->] (E) edge (D);
	\end{scope}
\end{tikzpicture}
\end{figure}

Now we reverse the graph and we get the following:
\begin{figure}[h]
\centering
\begin{tikzpicture}
	\begin{scope}[every node/.style = {circle, thick, draw},
		every label/.append style={font = \small}]
		\node (A) at (0,0) {A};
		\node (B) at (3,0) {B};
		\node (C) at (6,0) {C};
		\node (D) at (0,-2) {D};
		\node (E) at (3,-2) {E};
		\node (F) at (6,-2) {F};
		\node (G) at (9,-1) {G};
	\end{scope}
	\begin{scope}[>={Stealth[black]},
			every edge/.style={draw=black, very thick}]
		\path [<-] (A) edge[bend right=30] (B);
		\path [<-] (A) edge (D);
		\path [<-] (B) edge[bend right=30] (A);
		\path [<-] (B) edge (E);
		\path [<-] (C) edge (B);
		\path [<-] (C) edge (G);
		\path [<-] (F) edge (C);
		\path [<-] (F) edge (E);
		\path [<-] (G) edge (F);
		\path [<-] (E) edge (D);
	\end{scope}
\end{tikzpicture}
\end{figure}


Finally, we can loop through the nodes in postorder, starting with the node
that was close the last:
\[
	\mathrm{postorder}=\{C,G,F,A,B,E,D\}
\]
\newpage
\begin{enumerate}
\def \itemsep{0pt}
	\item We run DFS on node $C$ and we get the SCC:
		\[
			s_1=\{C,F,G\}
		\]
	\item We skip nodes $G$ and $F$ in the list since they are already in
		a SCC and we run DFS on node $A$, getting the SCC:
		\[
			s_2=\{A,B\}	
		\]

	\item We skip node $B$ since it is already in a SCC and we 
		run DFS on node $E$, getting the SCC:
		\[
			s_3=\{E\}
		\]
	\item We finally run DFS on node $D$, and we get the last SCC:
		\[
			s_4=\{D\}
		\]
\end{enumerate}
Now, returning to the initial graph, we have found the following SCCs:

\begin{figure}[h]
\centering
\begin{tikzpicture}
	\begin{scope}[every node/.style = {circle, thick, draw},
		every label/.append style={font = \small}]
		\node (A) at (0,0) {A};
		\node (B) at (3,0) {B};
		\node (C) at (6,0) {C};
		\node (D) at (0,-2) {D};
		\node (E) at (3,-2) {E};
		\node (F) at (6,-2) {F};
		\node (G) at (9,-1) {G};
	\end{scope}
	\begin{scope}[>={Stealth[black]},
			every edge/.style={draw=black, very thick}]
		\path [->] (A) edge[bend right=30] (B);
		\path [->] (A) edge (D);
		\path [->] (B) edge[bend right=30] (A);
		\path [->] (B) edge (E);
		\path [->] (C) edge (B);
		\path [->] (C) edge (G);
		\path [->] (F) edge (C);
		\path [->] (F) edge (E);
		\path [->] (G) edge (F);
		\path [->] (E) edge (D);
	\end{scope}
	\draw[rounded corners=15pt] (-1,-0.75) rectangle ++(5,1.5);
	\draw[rounded corners=15pt] (-1,-2.75) rectangle ++(2,1.5);
	\draw[rounded corners=15pt] (2,-2.75) rectangle ++(2,1.5);
	\draw[rounded corners=15pt] (5,-2.75) rectangle ++(5,3.5);

\end{tikzpicture}
\end{figure}
\vspace{-5 pt}
\subsubsection*{Explanation}
Let's look at why this algorithm works. 

Firstly, we have to take into account that the SCCs of a graph $G$ are preserved when 
we reverse all the edges and get $G^t$. The only relevant issue is the order in which
we process the SCCs. Let's assume we have a graph that has two SCC's:
\begin{figure}[h!]
	\centering
	\begin{tikzpicture}
	\path[draw] plot [smooth cycle,tension=1]  coordinates 
		{(0,0) (1.5,1) (2,0) (0.5,-1)};
	\node (p1) at (1,0){$c_1$};
	
	\path[draw] plot [smooth cycle,tension=1]  coordinates 
		{(3,0) (4.5,1) (5,0) (3.5,-1)};
	\node (p1) at (4,0){$c_2$};

	\begin{scope}[>={Stealth[black]},
			every edge/.style={draw=black}]
		\path[->, ] (2,0) edge (3,0);
	\end{scope}

	\end{tikzpicture}
\end{figure}
We have three options:
\begin{itemize}
	\def\itemsep{0pt}
	\item \emph{They are not connected}. In this case, it does not matter
		whether we explore one or the other first since DFS will not 
		\say{leak} from one of them to the other
	\item \emph{They are connected only in one direction}. This is 
		the case that the figure shows and it is the most important one.
		If we explore $c_2$ before exploring $c_1$, DFS will first
		explore the entirety of $c_2$ and then, when it is exploring
		$c_1$ it will not \say{leak} to $c_2$ because those nodes are already
		marked as visited.
	\item \emph{They are connected in both directions}. This can never happen
		since, if they were to be connected bidirectionally, they would 
		form a single SCC, not two.
\end{itemize}
\newpage
Therefore, the correctness of this algorithm simply depends on exploring the SCCs in the 
right order.

Let's demonstrate the following claim: 
\vspace{-5pt}
\begin{center}
	\itshape
	In the previous setting, the maximum closing time of the nodes in $c_1$ will
	be greater than the maximum closing time of the nodes in $c_2$
\end{center}
We just have to distinguish two cases:
\begin{itemize}
	\item
	If we started exploring $c_2$ before $c_1$, there is no way to get to $c_1$ from
	$c_2$. Therefore, we will start exploring $c_1$ when we have already closed $c_2$,
	which means that all nodes in $c_2$ will be closed before the nodes in $c_1$
	are even \say{opened}.
	\item
	If we started exploring $c_1$ before $c_2$, at some point, DFS will leak to 
	$c_2$ and it will explore it entirely before returning to $c_1$. Therefore,
	the closing time of every node in $c_2$ will be lower than the closing time
	of the node where DFS was started in $c_1$.
\end{itemize}
This completes the proof of that statement. Let's now apply it to check the correctness 
of this algorithm using induction. 

\begin{itemize}
	\item \textit{Base case:}

After reversing the graph, we start with the node that was closed the last ($a$). Let's call
it's SCC $s_1$. Let's assume that there is an edge from $s_1$ to another SCC ($s_\alpha$) in
$G^t$, which would make DFS leak to that SCC. However, if that was the case, there
would be an edge from $s_\alpha$ to $s_1$ in $G$ and, therefore, we can apply the 
previous claim. 

In that scenario, the maximum closing time of  $s_\alpha$ would be greater than the maximum
closing time of $s_1$. This is a contradiction because we have stated that $a\in s_1$ 
is the node that was closed the last.

Therefore, there cannot be any edge from $s_1$ to another SCC.

\item \textit{Inductive step:}

	Let's now assume that we have already processed $n$ SCCs. We now choose a node $b$,
	which is the node with the highest closing time such that $b \not \in s_i, \
	i=1,\dots n$. 
	
	We will call the SCC of this node $s_{n+1}$.
	As before, we want to prove that $s_{n+1}$ is not 
	connected to any $s_\beta$ that hasn't been explored yet in $G^t$.

	We will assume that there is an edge from $s_{n+1}$ to a $s_{\beta}\not \in 
	\{s_1\dots s_n\}$. Therefore, in $G$, there is an edge from $s_{\beta}$ to
	$s_{n+1}$, which implies that the maximum closing time of $s_{\beta}$ is higher than
	the maximum closing time of $s_{n+1}$. 

	Let's define $x_\beta:=$( the node with the maximum closing time of $s_\beta$ ). 
	We just stated that the closing time of $x_\beta$ is greater than the closing time 
	of $b$ (there cannot be a node with a greater closing time in $s_{n+1}$).

	However, this is a contradiction. Therefore, $s_{n+1}$ cannot
	be connected to an $SCC$ that has not been explored and, thus, DFS will not leak.
	\\ \null \hfill $\qedsymbol$
\end{itemize}

\newpage
\cppcode[firstline=20]{code/Kosaraju.cpp}
\noindent \textbf{\boldmath Running time: $\mathcal{O}(V+E)$}
\\ {\small (V = vertices, E = edges)}

\subsubsection*{Remarks}
\begin{itemize}
	\item The SCCs of any graph form a DAG
\end{itemize}

\newpage
\section{Kruskal's}
Kruskal's algorithm finds a minimum spanning forest of the given graph. To do so, it
uses UFDS to keep track of which nodes are already connected.
\subsection{UFDS}
UFDS (Union-find data structure) is a data structure that stores a partition of 
the vertices into sets such that all vertices in the same set are connected.
Each node $u$ will have two properties:
\begin{itemize}
	\setlength\itemsep{0pt}
	\item \textit{Parent}. It is the representative of the partition that
		contains $u$.
	\item \textit{Rank}. It is an upper bound of the height of the \say{tree}
		that starts on $u$. 
\end{itemize}
\cppcode[firstline=20]{code/ufds.cpp}
\noindent \textbf{\boldmath Running time: $\mathcal{O}(E\log V)$}
\\ {\small (V = vertices, E = edges)}
\subsubsection{Explanation}
There are two main optimizations that are applied in this implementation:
\begin{itemize}
	\item \textit{Union by rank}. When we join two partitions, we have to choose
		a node to represent the new partition. In order to choose between the 
		two parents, we use their rank. 

		Our goal is to minimize the height of the trees that start at every node.
		Therefore, we pick the node with the highest rank as the parent. This 
		choice ensures that the rank of both parents will not increase. 

		However, if both parents have the same rank, we can choose either of 
		them.
	\item \textit{Path shortening}. The \texttt{find} function updates the value 
		of the parent of each node so that it does not point to it's 
		\say{immediate} parent but to the highest possible ancestor. This
		difference increases the efficiency of subsequent executions of the 
		\texttt{find} routine. 
\end{itemize}

\subsection{Kruskal's}
Using UFDS, the implementation of Kruskal is trivial. We just edges that connect 
vertices that are not connected already until all vertices are connected.
\cppcode[firstline=20]{code/kruskal.cpp}
\noindent \textbf{\boldmath Running time: $\mathcal{O}(\max\{E,V\}\log(\max\{E,V\}) )$}
\\ {\small (V = vertices, E = edges)}

\newpage
\section{Flows and cuts}
In this section we will consider graphs as networks of pipes and the \say{weights}
of the edges will be their capacity.

\subsection{Edmonds-Karp's (Max Flow)}
The first problem that we have to consider is how to find the maximum flow from one
source node to a sink node. Edmonds-Karp's is an implementation of Ford-Fulkerson's 
that solves this problem.

This algorithm has some major caveats: it requires the use of an adjacency matrix
and it is not as fast as other options.

To improve its running time, this implementation uses both an adjacency matrix 
and adjacency list. However, we have to be careful when creating the adjacency
list since for every forward edge, we need a backward edge that will start with 
a capacity of 0 but its capacity may increase. 

Therefore, the code for adding edges would be the following:
\begin{minted}{cpp}
// Add a directed edge
void addedgeUni(int orig, int dest, ll flow) {
    adjList[orig].pb(dest);
    adjMat[orig][dest] = flow;
    adjList[dest].pb(orig);   //Add edge for residual flow
}
// Add a bidirectional edge
void addEdgeBi(int orig, int dest, ll flow) {
    adjList[orig].pb(dest);
    adjList[dest].pb(orig);
    adjMat[orig][dest] = flow;
    adjMat[dest][orig] = flow;
}
\end{minted}
\newpage
\cppcode[firstline=20,lastline=63]{code/edmonds-karp.cpp}
\noindent \textbf{\boldmath Running time: $\mathcal{O}(VE^2)$}
\\ {\small (V = vertices, E = edges)}
\subsubsection{Remarks}
\begin{itemize}
	\item If we have multiple sources $s_1,\dots, s_n$ and multiple sinks 
		$t_1,\dots t_n$, we can use the same algorithm. We just have to create a 
		source $s$ that connects to $s_i$ with edges of infinite capacity and
		a sink $t$ such that $s_i$ is connected to $t$ with edges of infinite
		capacity.
		
	\item If we have vertex capacities, we can split the vertex $v$ into two 
			vertices: $v_{in}$ and $v_{out}$, connected with an edge that has 
			the vertex capacity as a capacity. 
			Then we connect the incoming edges to $v_{in}$ and the out-coming 
			edges to $v_{out}$


\end{itemize}
\section{Directed Acyclic Graphs (DAG)}
\subsection{Topological sort}
A topological sort of a directed graph is a linear ordering of its vertices
such that for every directed edge u $\rightarrow$ v, u comes before v in the
ordering.
A topological sorting is possible if and only if the graph has no directed
cycles, that is, it has to be a DAG. As an example:

\begin{figure}[h!]
\centering
\begin{subfigure}[]{0.6\textwidth}
\begin{center}
\begin{tikzpicture}
	\begin{scope}[every node/.style = {circle, thick, draw},
		every label/.append style={font = \small}]
		\node (A) at (0,0) {0};
		\node (B) at (2,0) {1};
		\node (C) at (2,-2) {2};
		\node (D) at (4,0) {3};
		\node (E) at (6,0) {4};
		\node (F) at (8,0) {5};
		\node (G) at (6,-2) {6};
		\node (H) at (8,-2) {7};
	\end{scope}
	\begin{scope}[>={Stealth[black]},
			every edge/.style={draw=black, very thick}]
		\path [->] (A) edge (B);
		\path [->] (A) edge (C);
		\path [->] (C) edge (D);
		\path [->] (B) edge (D);
		\path [->] (D) edge (E);
		\path [->] (C) edge [bend right=10] (F);
		\path [->] (H) edge (G);
	\end{scope}
\end{tikzpicture}
		
\end{center}
\end{subfigure}
\begin{subfigure}[]{0.35\textwidth}
\begin{center}
	\begin{tikzpicture}	
		\draw (0,0) pic[]{array_rep={}{0, 1, 2, 3, 4, 5, 7, 6}};
	\end{tikzpicture}
\end{center}
\end{subfigure}


\end{figure}

Regarding the algorithm to obtain the topological sorting, there 
are two alternatives:
\begin{itemize}
		\setlength{\itemsep}{0pt}
		\item A modified version of DFS 
		\item Kahn's algorithm
\end{itemize}
\newpage
\subsubsection*{Using DFS}
The key observation is:
\begin{center}
		\itshape
		When DFS closes a node, all nodes that are accessed from
		that node are already closed
\end{center}
Therefore if we order the vertices regarding the DFS closing time, we can be 
sure that if there exists an edge u $\rightarrow$ v, v will be placed before
u since it must have been closed before v was closed. Now we should
realize that to get the topological sorting, we only need to reverse
this ordering.
\cppcode[firstline=20]{code/topoSort1.cpp}
\noindent \textbf{\boldmath Running time: $\mathcal{O}(E+V)$}
\\ {\small(V = vertices, E = edges)}
\clearpage
\subsubsection*{Kahn's algorithm}
Kahn's algorithm is a very simple algorithm. It simply keeps track of the in-degree
of every vertex and only adds to the topological sort list vertices that do not have
any incident edges.

Furthermore, this algorithm can be easily adapted to store all possible topological 
orderings by using a recursive function and running through all possibilities 
at each step
\cppcode[firstline=20]{code/topoSort1.cpp}
\noindent \textbf{\boldmath Running time: $\mathcal{O}(E+V)$}
\\ {\small(V = vertices, E = edges)}


\newpage
\section{Bipartite Graphs}
A graph is bipartite if its  vertices can be split into two different so that
no two nodes in the same set are connected by an edge.
\subsubsection*{Definitions}
It is important to note that in this section a path 
(unless specified otherwise) will be a simple path (not containing repeated
vertices or edges).
\begin{itemize}
		\setlength{\itemsep}{2pt}
		\item \textit{Matching}. Set of pairwise non-adjacent edges in a graph.
				That is to say, no more than one edge from the set should be
				incident to any vertex of the graph.
		\item \textit{Saturated} (a vertex with respect to some matching). A
				saturated vertex is a vertex that is an endpoint of an edge
				contained in the matching.
		\item \textit{Alternating path }(with respect to some matching). Path
				in which the edges alternately belong / do not belong to the 
				matching.
		\item \textit{Augmenting path }(with respect to some matching $M$). 
				Alternating path whose initial and final vertices are 
				unsaturated (they do not belong to the matching).
				That is to say, if we represent the edges already
				included in $M$ as $I$ and the new ones as $U$, it
				follows this pattern:
				\[
					 U - I - U - \dots - U - I - U
			 	\]
				We can also notice that there will be 
				$\left\lfloor \frac{|P|}2 \right \rfloor$ edges already
				included and 
				$\left\lceil \frac{|P|}2 \right \rceil$ new edges.
		\item \textit{Independent set}. Set of vertices in which no
				two vertices represent an edge in the original graph.
		\item \textit{Vertex coverage}. Set of vertices $C$ that fulfills
				the following property: Every vertex of $G$ is incident on,
				at least, one vertex of $C$.
\end{itemize}
From these definitions, we can derive the following features in a bipartite
graph:
\begin{itemize}
		\item \textit{Maximum cardinality bipartite matching (MCBM)}. 
				Matching that contains the maximum number of edges.
		\item \textit{Maximum independent set (MIS)}. Independent set that
				fulfills that if any other vertex was to be added to it, there
				would be an edge between two of the vertices of the set.
		\item \textit{Minimum vertex coverage (MVC)}. Vertex coverage with the 
				minimum number of vertices.
\end{itemize}
Furthermore, it can be proved that even though these \say{sets} may not be
unique, their cardinality is. Additionally, they are closely related among 
them:
\[
		\mathrm{|MIS| = |V| - |MCBM| }
		\qquad 
		\qquad 
		\mathrm{|MCBM| = |MVC|}
\]
\newpage
\subsubsection*{Example}
\begin{wrapfigure}{r}{0.4\textwidth}
		\vspace{-30pt}
\centering
\begin{tikzpicture}
	\begin{scope}[every node/.style = {circle, thick, draw},
		every label/.append style={font = \small}]
		\node (A) at (0,0) {1};
		\node (B) at (0,-2) {2};
		\node (C) at (3,0) {3};
		\node (D) at (3,-2) {4};
		\node (E) at (3,-4) {5};
	\end{scope}
	\begin{scope}[>={Stealth[black]},
			every edge/.style={draw=black, very thick}]
		\path [-] (A) edge[draw=red] (C);
		\path [-] (A) edge (D);
		\path [-] (A) edge (E);
		\path [-] (B) edge (C);
		\path [-] (B) edge [draw=red](D);
		\path [-] (B) edge (E);
	\end{scope}
	\path [draw=black,dashed, rounded corners=10pt] (2,-5) rectangle (4, 1);
	\path [draw=black,dashed, rounded corners=10pt] (-1,-3) rectangle (1, 1);
	\node [anchor=south] at (0,1) {MVC};
	\node [anchor=south] at (3,1) {MIS};
\end{tikzpicture}
\vspace{-40pt}
\end{wrapfigure}
In this example we have a bipartite graph where nodes 1 and 2 belong to the 
first set of vertices and nodes 3, 4 and 5 to the second set.
\begin{itemize}
		\item Since there are only two vertices in the first
				set, the MCBM has to be at most two. We have found
				a matching of size two (in red), so that must be
				the solution.
		\item The MIS has to be one of the two sets of vertices since all
				vertices from one set are connected to all vertices from the
				second set. We pick the maximum one. 
		\item Finally, the MVC has to be the smallest set for the same reasons.
\end{itemize}
Finally, we can also notice how the MCBM is not unique even though its cardinality is 
unique.
\subsection*{Berge's lemma}
\textit{A matching $M$ is maximum $\iff$ there is no augmenting path relative to the
		matching $M$}
\subsubsection{Proof}
We can prove it by contradiction:
\begin{itemize}
		\item [$\implies$]

				Let there be an augmenting path $P$ relative to $M$. This path will
				necessarily be of odd length (it has one more edge not in $M$ than
				in $M$). We can now create a new matching 
				$M'=M\oplus P=(M\setminus P) \cup (P\setminus M)$.

				Now we have to see that this matching is valid. In order for it not to
				be valid, there must be a vertex such that two edges in $M'$ are
				adjacent to. Furthermore, one of those edges must belong to 
				$P\setminus M$ while the other one belongs to $M$. However, the second
				one must then also belong to $P$ ($P$ alternates edges) and will not
				be in $M\setminus P$. Therefore, it cannot  be in $M'$ and we reached
				a contradiction

				Finally, we can see that there are $|M|+1$ edges in $M'$ since:
				\begin{itemize}
						\setlength{\itemsep}{0pt}
						\item By subtracting $P$ from $M$ we delete 
								$\left \lfloor\frac {|P|}2\right\rfloor $ edges
						\item By adding $P\setminus M$, we add 
								$\left \lceil\frac {|P|}2\right\rceil $ edges

				\end{itemize}
				Therefore, $M$ could not be a maximum matching
		\item [$\impliedby$]
				Let there be a matching $M'$  of greater cardinality than $M$.
				We can consider the subgraph $Q=M\oplus M'$. Any vertex in 
				this subgraph has a maximum degree of 2, which means that all
				connected components in it are either:
				\begin{itemize}
						\setlength{\itemsep}{0pt}
						\item An isolated vertex
						\item a (simple)path whose edges are alternate from $M$ 
								and $M'$
						\item A cycle of even length whose edges are 
								alternate from $M$ and $M'$
				\end{itemize}
				However, since there are more edges from $M'$ than from $M$, 
				there must be at least one component with more edges from $M'$
				than from $M$. Since such path must be alternating, it will
				have initial and final vertices unsaturated by $M$, making
				it an augmenting path for $M$.
\end{itemize}
\hfill $\qedsymbol$
\subsection{Algorithms to find MCBM}
\subsubsection{Reduce to Max-Flow}
The bipartite matching problem can be easily reduced to a max-flow problem.
We only have to:
\begin{itemize}
		\setlength{\itemsep}{0pt}
		\item Make the edges between the two components directed from the 
				first component to the second (all edges must go in the 
				same direction)
		\item Connect the source to all nodes in the first component.
		\item Connect the nodes in the second component to the sink.
		\item Set the capacities of all edges to 1.
\end{itemize}

\noindent
For instance:
\begin{figure}[h!]
\centering
\begin{subfigure}[b]{0.45\textwidth}
\centering
\begin{tikzpicture}
	\begin{scope}[every node/.style = {circle, thick, draw},
		every label/.append style={font = \small}]
		\node (A) at (0,0) {1};
		\node (B) at (0,-2) {2};
		\node (C) at (0,-4) {3};
		\node (D) at (3,0) {4};
		\node (E) at (3,-2) {5};
		\node (F) at (3,-4) {6};
	\end{scope}
	\begin{scope}[>={Stealth[black]},
			every edge/.style={draw=black, very thick}]
		\path [-] (A) edge (D);
		\path [-] (A) edge (E);
		\path [-] (A) edge (F);
		\path [-] (B) edge (D);
		\path [-] (B) edge (E);
		\path [-] (B) edge (F);
		\path [-] (C) edge (F);
	\end{scope}
	\node[anchor = south] at (1.5,0.5) {\small \bfseries Original graph:};
\end{tikzpicture}
\end{subfigure}
\hfill
\begin{subfigure}[b]{0.5\textwidth}
\centering
\begin{tikzpicture}
	\begin{scope}[every node/.style = {circle, thick, draw},
		every label/.append style={font = \small}]
		\node (A) at (0,0) {1};
		\node (B) at (0,-2) {2};
		\node (C) at (0,-4) {3};
		\node (D) at (3,0) {4};
		\node (E) at (3,-2) {5};
		\node (F) at (3,-4) {6};
		\node (Sc) at (-2, -2){S};
		\node (Sk) at (5, -2){T};
	\end{scope}
	\begin{scope}[>={Stealth[black]},
			every edge/.style={draw=black, very thick}]
		\path [->] (A) edge (D);
		\path [->] (A) edge (E);
		\path [->] (A) edge (F);
		\path [->] (B) edge (D);
		\path [->] (B) edge (E);
		\path [->] (B) edge (F);
		\path [->] (C) edge (F);
		\path [->] (Sc) edge (A);
		\path [->] (Sc) edge (B);
		\path [->] (Sc) edge (C);
		\path [->] (E) edge (Sk);
		\path [->] (F) edge (Sk);
		\path [->] (D) edge (Sk);
	\end{scope}
	\node[anchor = south] at (1.5,0.5) {\small \bfseries New graph:};
\end{tikzpicture}
\end{subfigure}
\end{figure}

After this process, we can run the normal max-flow algorithm and the maximum
flow between the source and the sink will be the result. This will have a 
time complexity of {\boldmath $\mathcal{O}(VE^2)$} using Edmonds-Karp's
or {\boldmath $\mathcal{O}(V^2E)$} using Dinic's

\newpage
\subsubsection{Kuhn's Algorithm}
On a high level, the algorithm implements the following pseudocode 
(where M is the matching):
\begin{minted}[escapeinside=||,mathescape]{text}
M <- |$\varnothing$|
while (able to find augmenting path P){
	M <- P |$\oplus$| M
}
\end{minted}
This is a direct application of Berge's lemma; however, if we were to 
implement it naïvely, we could end up with a time complexity of 
$\mathcal{O}(V^2E)$ since there will be at most $V$ augmenting paths
and to find each of them, we would need to check for paths that start 
on each node and then run DFS to find the actual path.
To make it faster, it can be implemented using the following 
pseudocode (where $n_1$ is the size of the first set in the graph):
\begin{minted}[escapeinside=||, mathescape]{text}
M <- |$\varnothing$|
for (i = 0 ... n1){
	P <- augmenting path that starts on i
	if(P != |$\varnothing$|){
		M <- P |$\oplus$| M
	}
}
\end{minted}
To understand why this implementation should work, it is necessarily to
make a few observation:
\begin{itemize}
		\setlength{\itemsep}{2pt}
		\item The augmenting path found in each iteration starts on a 
				vertex in the first set, then traverses some number
				of vertices that are already saturated and it ends
				on a vertex of the second set that is not saturated.
		\item When we update M with the new P, no vertices that were
				saturated before will be unsaturated now.  Since
				all saturated vertices in P are connected by two 
				adjacent edges, one of them will still be present 
				in P $\oplus$ M since it could not be present on M before.
		\item If there is no augmenting path that starts from i in the 
				current iteration, it will be impossible to find one 
				in a some future iteration. This is because no vertex 
				can go from being saturated to being unsaturated in this
				process.
\end{itemize}
With all of this in mind, we can obtain a c++ implementation that runs on 
$\mathcal{O}(EV)$. Furthermore, if we make sure that the first set of the
bipartite graph is the one with less elements, the runtime will be faster
since the iterations of the outer loop will decrease.

As we can see, the search for an augmenting path from a node will try
all options to pair that node to a node in the second set. For each one
of them, if they are already matched to some node in the first set, 
it will try to create an augmenting path from there. The algorithm will
stop when it finds an unsaturated vertex or runs out of vertex to check.
\newpage
\cppcode[firstline=20]{code/bipartite_Kuhn.cpp}
\noindent \textbf{\boldmath Running time: $\mathcal{O}(EV)$}
\\ {\small(V = vertices, E = edges)}






