\documentclass[12pt]{report}
\usepackage[usenames,dvipsnames]{xcolor}
\usepackage{tcolorbox}
\usepackage[margin=1in]{geometry} 
\usepackage{amsmath,amsthm,amssymb}
\usepackage[english]{babel}
\usepackage[T1]{fontenc} %escribe lo del teclado
\usepackage[utf8]{inputenc} %Reconoce algunos símbolos
\usepackage{graphicx}
\usepackage{multirow}
\usepackage{wrapfig}
\usepackage{diagbox} %diagonal cells
\usepackage{dirtytalk}
\usepackage{minted}
\usepackage{booktabs}
\usepackage{import}
\usepackage{hyperref}
\usepackage{titlesec} % To configure subsections

\titleformat*{\subsection}{\normalsize\bfseries}
\setcounter{tocdepth}{1} % Only sections
\definecolor{bg}{RGB}{240,240,240}
\newcolumntype{Z}{>{\ttfamily}{c}<{}}


\definecolor{mintedbackground}{rgb}{0.95,0.95,0.95}

\newmintedfile[cppcode]{cpp}{
bgcolor=mintedbackground,
linenos=true,
numberblanklines=true,
numbersep=5pt,
gobble=0,
frame=leftline,
framerule=0.4pt,
framesep=2mm,
funcnamehighlighting=true,
tabsize=4,
obeytabs=false,
mathescape=false
samepage=true, %with this setting you can force the list to appear on the same page
showspaces=false,
showtabs =false,
texcl=false,
fontsize=\small
}

\setminted[cpp]{
bgcolor=mintedbackground,
linenos=true,
numberblanklines=true,
numbersep=5pt,
gobble=0,
frame=leftline,
framerule=0.4pt,
framesep=2mm,
funcnamehighlighting=true,
tabsize=4,
obeytabs=false,
mathescape=false
samepage=true, %with this setting you can force the list to appear on the same page
showspaces=false,
showtabs =false,
texcl=false,
fontsize=\small
}

\begin{document}

\begin{titlepage}
    \begin{center}
        \vspace*{1cm}
 
        \Huge
        \textbf{CP implementations}
 

 
        \vspace{1.5cm}
 
        \textbf{David del Val}
 
        \vfill
 
 
        \vspace{0.8cm}
 
 
        \Large
	\today
 
    \end{center}
\end{titlepage}

\tableofcontents 

\chapter{Graphs}

\section{Dijkstra}
Shortest path from \texttt{orig} node to \texttt{dest} (or to every node) in a graph
that does not contain negative edges. 
It chooses the best path greedily in each iteration and, therefore, it only works
on graphs without negative weights. 
\cppcode[firstline=20]{code/graph/dijkstra.cpp}
\noindent \textbf{\boldmath Running time: $\mathcal{O}(V+E\log(E))$}
\\ {\small(V = vertices, E = edges)}

\subsection*{Observations}
\begin{itemize}
	\item If we ignore the check in line 30, we can return the distances 
		vector, which will contain the shortest distance from \texttt{dist}
		to every other node.
	\item If we are doing some kind of pruning it is imperative that we prune 
	as many branches as possible in the main loop. That is to say, we should
	introduce as many \texttt{if} statements in line 36 to make sure that we
	run the \texttt{for} loop as few times as possible. 

	An example of this approach is problem \texttt{UVA-11635}. In that problem, we
	add a lot of branches to the queue (we may run the \texttt{for} loop
	twice in some nodes) but we prune them in the main loop. Thus the 
	running time is still acceptable.

\end{itemize}

\section{Bellman Ford}
Shortest parth from \texttt{orig} to every other node. It is slower than Dijkstra but 
it works on graphs with negative weights. 

This algorithm works by trying to relax every edge $V-1$ times. If there are no 
negative cycles, after $V-1$ iterations, we must have found the minimum distance
to every node. Therefore if after these iterations, we run another
iteration and the distance to a node decreases, we must have a negative cycle.

\cppcode[firstline=20]{code/graph/bellman_ford.cpp}
\noindent \textbf{\boldmath Running time: $\mathcal{O}(VE)$}
\\ {\small(V = vertices, E = edges)}
\subsection*{Observations}
\begin{itemize}
	\item If we keep track of the distance that decrease when we check for
		a negative cycle, we will get at least one node of 
		every negative cycle present in the graph.

		We can use this, for instance, to check if we can reach a node with 
		a cost smaller than a given bound. If it is connected to a node in a 
		negative cycle, it's distance will be as small as we want it to be
		(by looping in the cycle).

		This can be seen at play in \texttt{UVA-10557}
	\item If we modify slightly the main loop, iteration $i$ will be the 
		result of considering paths of at most $i+1$ edges:
		\begin{minted}{cpp}
for (int i = 0; i < n - 1; ++i) {
        for (auto e : edges) {
            dists2[e.fi.se] = min(dists2[e.fi.se], dists[e.fi.fi] + e.se);
        }
	dists = dists2;
}
		\end{minted}
		This can be seen at play in \texttt{UVA-11280}
		

\end{itemize}	

\end{document}
