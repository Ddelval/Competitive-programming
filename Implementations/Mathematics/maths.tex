%! TEX root = ../implementations.tex
\chapter{Mathematics}

\section{General calculations}
\subsection{Binary exponentiation}
Binary exponentiation calculates a power in logarithmic 
time. Furthermore, it can be used for modular arithmetic:
\cppcode[firstline=20,]{code/binary_exp.cpp}
\noindent \textbf{\boldmath Running time: $\mathcal{O}(\log(\mathrm{exp}))$}


\newpage
\section{Modular arithmetic}
\subsection{Inverses}
To calculate a modular inverse, we will use Fermat's little theorem:
\[
	a^{p-1}\equiv 1 \ \mathrm{mod} \ p  \ \implies \  a^{p-2}\equiv a^{-1} \ \mathrm{mod} p
\]
If we combine this fact with binary exponentiation, we can obtain the 
modular inverse in logarithmic time:
\begin{minted}{cpp}
ll inverse(ll num) {
    return power(num, mod - 2);
}
\end{minted}
\noindent \textbf{\boldmath Running time: $\mathcal{O}(\log(\mathrm{mod}))$}

\section{Catalan numbers}
We define the nth Catalan number as:
\[
	C_n= \frac{1}{n+1}{2n\choose n} = \frac{1}{n+1}\frac{(2n)!}{n! \;n!}
\]
We can also define them recursively:
\[
	C_0 = 1 \qquad C_{n+1} = \sum_{i=0}^n\big (C_i \; C_{n-i}\big)
\]
They can be used to solve many different problems. For instance:
\begin{itemize}
	\item Number of different binary trees of $n$ nodes. We can look
		at the specific case $n=3$. As we can see in the figure below,
		$C_3=5$. Furthermore, we can clearly identify the recursive relationship:
		\[
			C_3= (\text{3 is root}) + (\text{2 is root}) + (\text{1 is root}) = C_2\cdot C_0 + 
			C_1\cdot C_1 + C_0\cdot C_2
		\]

		\begin{figure}[h!]
			\centering
			\scalebox{0.5}{
			\begin{tikzpicture}
				\begin{scope}[every node/.style = {circle, thick, draw},
					every label/.append style={font = \small}]
					\node (A) at (0,0) {3};
					\node (B) at (-1,-1) {2};
					\node (C) at (-2,-2) {1};
				\end{scope}
				\begin{scope}[>={Stealth[black]},
						every edge/.style={draw=black, very thick}]
					\path [-] (A) edge (B);
					\path [-] (B) edge (C);
				\end{scope}
			\end{tikzpicture}
			\begin{tikzpicture}
				\begin{scope}[every node/.style = {circle, thick, draw},
					every label/.append style={font = \small}]
					\node (A) at (0,0) {3};
					\node (B) at (-1,-1) {1};
					\node (C) at (-0.25,-2) {2};
				\end{scope}
				\begin{scope}[>={Stealth[black]},
						every edge/.style={draw=black, very thick}]
					\path [-] (A) edge (B);
					\path [-] (B) edge (C);
				\end{scope}
			\end{tikzpicture}

			\hspace*{60 pt}

			\begin{tikzpicture}
				\begin{scope}[every node/.style = {circle, thick, draw},
					every label/.append style={font = \small}]
					\node (A) at (0,0) {2};
					\node (B) at (1,-1) {3};
					\node (C) at (-1,-1) {1};
				\end{scope}
					\node (J) at (-1,-2) {};
				\begin{scope}[>={Stealth[black]},
						every edge/.style={draw=black, very thick}]
					\path [-] (A) edge (B);
					\path [-] (A) edge (C);
				\end{scope}
			\end{tikzpicture}
			\hspace*{30 pt}
			\begin{tikzpicture}
				\begin{scope}[every node/.style = {circle, thick, draw},
					every label/.append style={font = \small}]
					\node (A) at (0,0) {1};
					\node (B) at (1,-1) {2};
					\node (C) at (2,-2) {3};
				\end{scope}
					\node (J) at (-1,-2) {};
				\begin{scope}[>={Stealth[black]},
						every edge/.style={draw=black, very thick}]
					\path [-] (A) edge (B);
					\path [-] (B) edge (C);
				\end{scope}
			\end{tikzpicture}
			\begin{tikzpicture}
				\begin{scope}[every node/.style = {circle, thick, draw},
					every label/.append style={font = \small}]
					\node (A) at (0,0) {1};
					\node (B) at (1,-1) {3};
					\node (C) at (0.25,-2) {2};
				\end{scope}
					\node (J) at (-1,-2) {};
				\begin{scope}[>={Stealth[black]},
						every edge/.style={draw=black, very thick}]
					\path [-] (A) edge (B);
					\path [-] (B) edge (C);
				\end{scope}
			\end{tikzpicture}
	}
	\end{figure}
	\item Number of expressions that contain $n$ pairs of parenthesis correctly
			paired. For instance, for $n=3$, there only $C_3=5$ valid 
			arrangements are:
			\[
					()()()\quad ()(()) \quad (())() \quad ((())) \quad (()())
			\]
			\newpage
	\item Number of different ways of arranging $n+1$ factors in parenthesis
			In the case of $n=3$, we can use the factors $a, b, c, d$. Then,
			we will have the following options:
			\[
					(ab)(cd) \quad a(b(cd)) \quad ((ab)c)d \quad (a(bc))d \quad
					a((bc)d)
			\]
	\item Number of ways to split a convex polygon of $n+2$ sides 
			into triangles. 

			\begin{figure}[h!]
					\centering
			\resizebox{80 mm}{!}{\subimport{figures/}{cat1.pdf_tex}}
			\end{figure}

	\item Number of paths from the bottom-left corner to the top-right corner
			of a $n\times n$ grid moving only up or right at every point and
			while not reaching the diagonal.
			\begin{figure}[h!]
					\centering
			\resizebox{80 mm}{!}{\subimport{figures/}{cat2.pdf_tex}}
			\end{figure}
			
\end{itemize}
\newpage
\section{Permutations}
\subsection{Converting one permutation into another }
In this problem we are given two permutations of the same size 
and we are tasked with converting one of them into the other. 
In order to do so, we will swap the elements at two positions
in the permutation zero or more times. 

\begin{figure}[h!]
\centering
\begin{tikzpicture}
		\draw (0,0) pic[]{array_rep={A =}{1,2,3,4,5,7,6}};
		\draw (0,-1.25) pic[]{array_rep={B =}{1,3,2,4,5,7,6}};
\end{tikzpicture}

\end{figure}
In this case, the most efficient way is to swap positions 1 and 2 
of permutation B to get permutation A. From examples like this one, we can 
deduce the following observations:
\begin{itemize}
		\item One swap will fix at most two numbers out of
				place.
		\item We will need at most $n-1$ swaps to change one
				permutation into the other. Note that
				we can \say{skip} the swap that corresponds
				to the last element since, at that point, by
				the Pigeonhole Principle, all elements must be in 
				the correct position.
\end{itemize}

To solve the problem, we shall change the representation 
to a graph where the edge \texttt{a \nolinebreak$\rightarrow$ \nolinebreak b} 
represents that there exists a position \texttt{j} such that 
\texttt{A[j]=a, B[j]=b}. Since the previous example was extraordinarily 
simple, we will present a new one to show this process:

\begin{figure}[h!]
\centering
\begin{subfigure}[c]{0.4\textwidth}
\centering
\begin{tikzpicture}
		\draw (0,0) pic[]{array_rep={A =}{1,2,3,4,5,6,7}};
		\draw (0,-1.25) pic[]{array_rep={B =}{2,3,1,4,7,5,6}};
\end{tikzpicture}
		
\end{subfigure}
\begin{subfigure}[c]{0.4\textwidth}
\centering
\begin{tikzpicture}
	\begin{scope}[every node/.style = {circle, thick, draw},
		every label/.append style={font = \small}]
		\node (A) at (0,0) {1};
		\node (B) at (2,0) {2};
		\node (C) at (4,0) {3};
		\node (D) at (6,0) {4};
		\node (E) at (0,-2) {5};
		\node (F) at (2,-2) {6};
		\node (G) at (4,-2) {7};
	\end{scope}
	\begin{scope}[>={Stealth[black]},
			every edge/.style={draw=black, very thick}]
		\path [->] (A) edge (B);
		\path [->] (B) edge (C);
		\path [->] (C) edge[bend right=30] (A);
		\path [->] (D) edge[loop right, out=150, in=210, looseness=7] (D);
		\path [->] (E) edge[bend right] (G);
		\path [->] (F) edge (E);
		\path [->] (G) edge (F);
	\end{scope}
\end{tikzpicture}
\end{subfigure}
\end{figure}
This graph will fulfill the following properties:
\begin{itemize}
		\setlength{\itemsep}{2pt}
		\item All edges are part of a cycle (possibly of length 1). 
				This is due to the fact that all vertices have an 
				in-degree and out-degree of 1.
		\item Since all edges form cycles,  connectivity and
				strong connectivity are equivalent. In particular, 
				we can make the edges undirected without altering 
				the connectivity.
\end{itemize}

The final piece of the puzzle is noticing that the elements that 
form a cycle of length $k$ need exactly $k-1$ swaps to be rearranged.
In order to prove this, we can use induction. 

If $k\le 2$, it is trivial. We can now assume that we have a cycle of
length $k$. When we swap the first two elements, we can either fix
one element that was out of place or two. However, if we had fixed
two, that would mean that those two were interchanged, in which
case they would form a 2-cycle and they could not be part of any
larger cycle. Therefore, we can only fix one element in this first
swap. After this swap, the cycle has length $k-1$ and, by
the inductive hypothesis, we will need $(k-1)-1$ moves to rearrange
all the elements. Thus, if we add the swap that we already performed,
we obtain the expected result of $k-1$ swaps.

To get the final formula, we observe that if there is a big cycle
that includes all elements, we would need $n-1$ swaps and 
every cycle that we introduce decreases the number of swaps 
required by one.
Therefore:
\[
		\text{\# Swaps} = n- \text{\# Components}
\]
To implement this formula we can construct the graph using undirected
edges and count the components applying UFDS

\cppcode[firstline=55,]{code/perm_inv.cpp}
\noindent \textbf{\boldmath Running time: $\mathcal{O}(n)$}

\subsection{Converting one permutation into another using adjacent swaps}
This is a very similar problem; however, now we can only swap elements
that are next to each other in the array. Furthermore, in this analysis,
we will reduce the problem to calculating the minimum number of adjacent
swaps required to sort a permutation. This is equivalent since we can use
a bijective application to \say{redefine} the order according to how the
elements are present in the first permutation.

The key observation in this case is the fact that the solution is the 
number of inversions in the permutation. We can define an inversion as
a pair of indices $(i,j)$ such that $i<j$ but $A_i > A_j$. We can easily
notice that (choosing optimally), each adjacent swap will fix exactly one 
inversion since it will fix the relative order between two elements.

Furthermore, this idea is surprisingly easy to implement since we can 
use a modified version of Merge Sort that counts the swaps that would
have been required to place an element in a specific position while
maintaining a log-linear complexity.

To do so, we can just count at each step how many position we had to move
the elements of the left half of the array to sort them.

\cppcode[firstline=21,]{code/perm_adj_inv.cpp}
\noindent \textbf{\boldmath Running time: $\mathcal{O}(n\log(n))$}









