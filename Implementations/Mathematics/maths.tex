%! TEX root = ../implementations.tex
\chapter{Mathematics}

\section{General calculations}
\subsection{Binary exponentiation}
Binary exponentiation calculates a power in logarithmic 
time. Furthermore, it can be used for modular arithmetic:
\cppcode[firstline=20,]{code/binary_exp.cpp}
\noindent \textbf{\boldmath Running time: $\mathcal{O}(\log(\mathrm{exp}))$}


\newpage
\section{Modular arithmetic}
\subsection{Inverses}
To calculate a modular inverse, we will use Fermat's little theorem:
\[
	a^{p-1}\equiv 1 \ \mathrm{mod} \ p  \ \implies \  a^{p-2}\equiv a^{-1} \ \mathrm{mod} p
\]
If we combine this fact with binary exponentiation, we can obtain the 
modular inverse in logarithmic time:
\begin{minted}{cpp}
ll inverse(ll num) {
    return power(num, mod - 2);
}
\end{minted}
\noindent \textbf{\boldmath Running time: $\mathcal{O}(\log(\mathrm{mod}))$}

\section{Catalan numbers}
We define the nth Catalan number as:
\[
	C_n= \frac{1}{n+1}{2n\choose n} = \frac{1}{n+1}\frac{(2n)!}{n! \;n!}
\]
We can also define them recursively:
\[
	C_0 = 1 \qquad C_{n+1} = \sum_{i=0}^n\big (C_i \; C_{n-i}\big)
\]
They can be used to solve many different problems. For instance:
\begin{itemize}
	\item Number of different binary trees of $n$ nodes. We can look
		at the specific case $n=3$. As we can see in the figure below,
		$C_3=5$. Furthermore, we can clearly identify the recursive relationship:
		\[
			C_3= (\text{3 is root}) + (\text{2 is root}) + (\text{1 is root}) = C_2\cdot C_0 + 
			C_1\cdot C_1 + C_0\cdot C_2
		\]
		\begin{figure}[h!]
			\centering
			\scalebox{0.5}{
			\begin{tikzpicture}
				\begin{scope}[every node/.style = {circle, thick, draw},
					every label/.append style={font = \small}]
					\node (A) at (0,0) {3};
					\node (B) at (-1,-1) {2};
					\node (C) at (-2,-2) {1};
				\end{scope}
				\begin{scope}[>={Stealth[black]},
						every edge/.style={draw=black, very thick}]
					\path [-] (A) edge (B);
					\path [-] (B) edge (C);
				\end{scope}
			\end{tikzpicture}
			\begin{tikzpicture}
				\begin{scope}[every node/.style = {circle, thick, draw},
					every label/.append style={font = \small}]
					\node (A) at (0,0) {3};
					\node (B) at (-1,-1) {1};
					\node (C) at (-0.25,-2) {2};
				\end{scope}
				\begin{scope}[>={Stealth[black]},
						every edge/.style={draw=black, very thick}]
					\path [-] (A) edge (B);
					\path [-] (B) edge (C);
				\end{scope}
			\end{tikzpicture}

			\hspace*{60 pt}

			\begin{tikzpicture}
				\begin{scope}[every node/.style = {circle, thick, draw},
					every label/.append style={font = \small}]
					\node (A) at (0,0) {2};
					\node (B) at (1,-1) {3};
					\node (C) at (-1,-1) {1};
				\end{scope}
					\node (J) at (-1,-2) {};
				\begin{scope}[>={Stealth[black]},
						every edge/.style={draw=black, very thick}]
					\path [-] (A) edge (B);
					\path [-] (A) edge (C);
				\end{scope}
			\end{tikzpicture}
			\hspace*{30 pt}
			\begin{tikzpicture}
				\begin{scope}[every node/.style = {circle, thick, draw},
					every label/.append style={font = \small}]
					\node (A) at (0,0) {1};
					\node (B) at (1,-1) {2};
					\node (C) at (2,-2) {3};
				\end{scope}
					\node (J) at (-1,-2) {};
				\begin{scope}[>={Stealth[black]},
						every edge/.style={draw=black, very thick}]
					\path [-] (A) edge (B);
					\path [-] (B) edge (C);
				\end{scope}
			\end{tikzpicture}
			\begin{tikzpicture}
				\begin{scope}[every node/.style = {circle, thick, draw},
					every label/.append style={font = \small}]
					\node (A) at (0,0) {1};
					\node (B) at (1,-1) {3};
					\node (C) at (0.25,-2) {2};
				\end{scope}
					\node (J) at (-1,-2) {};
				\begin{scope}[>={Stealth[black]},
						every edge/.style={draw=black, very thick}]
					\path [-] (A) edge (B);
					\path [-] (B) edge (C);
				\end{scope}
			\end{tikzpicture}
	}
	\end{figure}
\end{itemize}

