%! TEX root = ../implementations.tex
\chapter{Mathematics}

\section{General calculations}
\subsection{Binary exponentiation}
Binary exponentiation calculates a power in logarithmic 
time. Furthermore, it can be used for modular arithmetic:
\cppcode[firstline=20,]{code/binary_exp.cpp}
\noindent \textbf{\boldmath Running time: $\mathcal{O}(\log(\mathrm{exp}))$}


\newpage
\section{Modular arithmetic}
\subsection{Inverses}
To calculate a modular inverse, we will use Fermat's little theorem:
\[
	a^{p-1}\equiv 1 \ \mathrm{mod} \ p  \ \implies \  a^{p-2}\equiv a^{-1} \ \mathrm{mod} p
\]
If we combine this fact with binary exponentiation, we can obtain the 
modular inverse in logarithmic time:
\begin{minted}{cpp}
ll inverse(ll num) {
    return power(num, mod - 2);
}
\end{minted}
\noindent \textbf{\boldmath Running time: $\mathcal{O}(\log(\mathrm{mod}))$}

\section{Catalan numbers}
We define the nth Catalan number as:
\[
	C_n= \frac{1}{n+1}{2n\choose n} = \frac{1}{n+1}\frac{(2n)!}{n! \;n!}
\]
We can also define them recursively:
\[
	C_0 = 1 \qquad C_{n+1} = \sum_{i=0}^n\big (C_i \; C_{n-i}\big)
\]
They can be used to solve many different problems. For instance:
\begin{itemize}
	\item Number of different binary trees of $n$ nodes. We can look
		at the specific case $n=3$. As we can see in the figure below,
		$C_3=5$. Furthermore, we can clearly identify the recursive relationship:
		\[
			C_3= (\text{3 is root}) + (\text{2 is root}) + (\text{1 is root}) = C_2\cdot C_0 + 
			C_1\cdot C_1 + C_0\cdot C_2
		\]

		\begin{figure}[h!]
			\centering
			\scalebox{0.5}{
			\begin{tikzpicture}
				\begin{scope}[every node/.style = {circle, thick, draw},
					every label/.append style={font = \small}]
					\node (A) at (0,0) {3};
					\node (B) at (-1,-1) {2};
					\node (C) at (-2,-2) {1};
				\end{scope}
				\begin{scope}[>={Stealth[black]},
						every edge/.style={draw=black, very thick}]
					\path [-] (A) edge (B);
					\path [-] (B) edge (C);
				\end{scope}
			\end{tikzpicture}
			\begin{tikzpicture}
				\begin{scope}[every node/.style = {circle, thick, draw},
					every label/.append style={font = \small}]
					\node (A) at (0,0) {3};
					\node (B) at (-1,-1) {1};
					\node (C) at (-0.25,-2) {2};
				\end{scope}
				\begin{scope}[>={Stealth[black]},
						every edge/.style={draw=black, very thick}]
					\path [-] (A) edge (B);
					\path [-] (B) edge (C);
				\end{scope}
			\end{tikzpicture}

			\hspace*{60 pt}

			\begin{tikzpicture}
				\begin{scope}[every node/.style = {circle, thick, draw},
					every label/.append style={font = \small}]
					\node (A) at (0,0) {2};
					\node (B) at (1,-1) {3};
					\node (C) at (-1,-1) {1};
				\end{scope}
					\node (J) at (-1,-2) {};
				\begin{scope}[>={Stealth[black]},
						every edge/.style={draw=black, very thick}]
					\path [-] (A) edge (B);
					\path [-] (A) edge (C);
				\end{scope}
			\end{tikzpicture}
			\hspace*{30 pt}
			\begin{tikzpicture}
				\begin{scope}[every node/.style = {circle, thick, draw},
					every label/.append style={font = \small}]
					\node (A) at (0,0) {1};
					\node (B) at (1,-1) {2};
					\node (C) at (2,-2) {3};
				\end{scope}
					\node (J) at (-1,-2) {};
				\begin{scope}[>={Stealth[black]},
						every edge/.style={draw=black, very thick}]
					\path [-] (A) edge (B);
					\path [-] (B) edge (C);
				\end{scope}
			\end{tikzpicture}
			\begin{tikzpicture}
				\begin{scope}[every node/.style = {circle, thick, draw},
					every label/.append style={font = \small}]
					\node (A) at (0,0) {1};
					\node (B) at (1,-1) {3};
					\node (C) at (0.25,-2) {2};
				\end{scope}
					\node (J) at (-1,-2) {};
				\begin{scope}[>={Stealth[black]},
						every edge/.style={draw=black, very thick}]
					\path [-] (A) edge (B);
					\path [-] (B) edge (C);
				\end{scope}
			\end{tikzpicture}
	}
	\end{figure}
	\item Number of expressions that contain $n$ pairs of parenthesis correctly
			paired. For instance, for $n=3$, there only $C_3=5$ valid 
			arrangements are:
			\[
					()()()\quad ()(()) \quad (())() \quad ((())) \quad (()())
			\]
			\newpage
	\item Number of different ways of arranging $n+1$ factors in parenthesis
			In the case of $n=3$, we can use the factors $a, b, c, d$. Then,
			we will have the following options:
			\[
					(ab)(cd) \quad a(b(cd)) \quad ((ab)c)d \quad (a(bc))d \quad
					a((bc)d)
			\]
	\item Number of ways to split a convex polygon of $n+2$ sides 
			into triangles. 

			\begin{figure}[h!]
					\centering
			\resizebox{80 mm}{!}{\subimport{figures/}{cat1.pdf_tex}}
			\end{figure}

	\item Number of paths from the bottom-left corner to the top-right corner
			of a $n\times n$ grid moving only up or right at every point and
			while not reaching the diagonal.
			\begin{figure}[h!]
					\centering
			\resizebox{80 mm}{!}{\subimport{figures/}{cat2.pdf_tex}}
			\end{figure}
			
\end{itemize}
\newpage
\section{Permutations}
\subsection{Converting one permutation into another }
In this problem we are given two permutations of the same size 
and we are tasked with converting one of them into the other. 
In order to do so, we will swap the elements at two positions
in the permutation zero or more times. 

\begin{figure}[h!]
\centering
\begin{tikzpicture}
		\draw (0,0) pic[]{array_rep={A =}{1,2,3,4,5,5,6}};
		\draw (0,-1.25) pic[]{array_rep={B =}{1,3,2,4,5,5,6}};
\end{tikzpicture}

\end{figure}
In this case, the most efficient way is to swap positions 1 and 2 
of permutation B to get permutation A. From these examples, we can 
deduce the following observations:
\begin{itemize}
		\item One swap will fix at most two numbers out of
				place.
		\item We will need at most $n-1$ swaps to change one
				permutation into the other. Note that
				we can \say{skip} the swap that corresponds
				to the last element since, at that point, by
				the Pigeon Principle, all elements must be in 
				the correct position.
\end{itemize}

To solve the problem, we shall change the representation 
to a graph where the edge \texttt{a \nolinebreak$\rightarrow$ \nolinebreak b} 
represents that there exists \texttt{j} such that \texttt{A[j]=a, B[j]=b}.
When draw these edges, we notice that all edges form cycles (possibly of 
length 1).
Furthermore, the cost of fixing all the positions that comprise each
cycle will be the length of the cycle  minus 1.

Finally, since the graph is made out of cycles,  
we know that in this graph connectivity and strong connectivity 
are equivalent and we can ignore the direction of the edges. 
This reduces the problem to counting the number of connected components
and the result is:
\[
		\text{\# Swaps} = n- \text{\# Components}
\]
This can be simply implemented using UFDS:
\cppcode[firstline=55,]{code/perm_inv.cpp}








